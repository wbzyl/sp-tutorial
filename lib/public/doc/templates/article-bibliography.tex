% !TEX TS-program = xelatex
% !TEX encoding = UTF-8 Unicode

\documentclass[12pt]{article}
\usepackage{geometry}
\geometry{a4paper}

\usepackage{fontspec}
\usepackage{hyperref}
\usepackage{xunicode}
\usepackage{xltxtra}
\usepackage{url}  % do przełamywania url
\usepackage[polish]{polyglossia}

\bibliographystyle{unsrt}

\title{Bibliografia do przedmiotu\\,,Środowisko Programisty''}
\author{Włodzimierz Bzyl}
\date{20 października 2009}

\begin{document}
\maketitle
\tableofcontents

\section{Podstawy}\label{sec:basics}

W fontach maszynowych, ang. \emph{monospaced fonts},
wszystkie znaki mają taką samą  szerokość. Za pomocą takich 
fontów składane są listingi lub kod programów.

\section{Poziom średnio zaawansowany}\label{sec:intermediate}

W książce ,,\LaTeX\ Web Companion'' opisano
format \LaTeX\ i podstawowe pakiety do niego.

\section{Rzeczy zaawansowane}\label{sec:advanced}

DocBookXsl jest ciekawym projektem. Więcej na ten temat
można poczytać w~\cite{wiki.docbookxsl}.

Do konwersji dokumentu z formatu latex na html najlepszym narzędziem
jest~\cite[uwaga, system szuka nowego opiekuna]{Gurari.TeX4ht}.

O fontach maszynowych już było na stronie~\pageref{sec:basics}.

\bibliography{sp}

\end{document}
